%% This is a new latex template for poster creation
%% For conferences / research meetings etc.
%% Following IGN guidelines

%% This template is based on tikzposter
%% With custom style but still a lot of stuff
%% That needs to be handed-specified
%% Also the comments are sketchy and the english is poor

%% Tikzposter ->
%% Copyright (C) 2014 by Pascal Richter, Elena Botoeva, Richard Barnard, and Dirk Surmann
%%
%%
%%
%%
%% History:
%% 17/02/2018 - Creation
%% 15/01/2020 - Update for the 2020 edition
%% 12/03/2021 - Date change for 2021 edition
%%
%%
%% If you want to modify the style,
%% Please define your custom blocks etc.
%% In the attached files
%% The name says what is contained in each file.
%% And the layout themes file is what sums up your custom layout
%% By precising the block style, background style, color style etc. you need
%%
%%
%% 
%%
%% Also if you need the doc, it is available here:
%% - pdf. doc : https://mirror.hmc.edu/ctan/graphics/pgf/contrib/tikzposter/tikzposter.pdf
%% - poster example : https://www.sharelatex.com/templates/53332341910d975953dffdab/v/1/pdf?inline=true&name=Tikzposter%20(extended%20example)


\documentclass[a0paper]{tikzposter} %Options for format can be included here

%Choose Layout
\usetheme{Steph}

% Packages
\usepackage[T1]{fontenc}
\usepackage[utf8]{inputenc}
\usepackage{natbib}
\usepackage{amsmath}
\usepackage{calc}
\usepackage{graphicx}
\usepackage{amssymb}
\usepackage{relsize}
\usepackage{multirow}
\usepackage{rotating}
\usepackage{bm}
\usepackage{url}
\usepackage{multicol}
\usepackage{array}
\usepackage{subfigure}
\usepackage{xcolor}
\usepackage{tikz}
\usetikzlibrary{arrows,patterns, calc, arrows.meta}
\tikzset{>={Latex[width=3mm,length=5mm]}}

\renewcommand{\rmdefault}{phv}
\renewcommand{\sfdefault}{phv} 
\renewcommand{\labelitemi}{$\bullet$}

\newcommand{\compresslist}{
    \setlength{\itemsep}{1pt}
	\setlength{\parskip}{0pt}
	\setlength{\parsep}{0pt}
}

\definecolor{IGNVert}{RGB}{163, 210,  11}
\definecolor{IGNGris}{RGB}{159, 164, 168}
\definecolor{IGNGrisFonce}{RGB}{101, 105, 110}

\graphicspath{./images}


% Here specify title, author and institution
% Logos that are normally in "titlegraphic"
% Are specified after
\title{\textbf{The SubDense project}}
\author{Mouhamadou Ndim, Bénédicte Bucher, Ana-Maria Raymond, Juste Raimbault, Julien Perret}
\institute{LASTIG, Univ. Gustave Eiffel, IGN-ENSG}
\titlegraphic{}

% Layout of title and logos, some other possible logos are available in images folder
% Don't hesitate to modify the positions if it doesn't suit you
\makeatletter
\renewcommand\TP@maketitle{
    \hspace{-.12\textwidth}
	\begin{tabular}{lcl}
	%logo box 1
    	\begin{minipage}[b][.1\textheight][b]{.15\textwidth}
        	\includegraphics[scale=0.9]{./images/LOGO_IGN}
          %\includegraphics[scale=0.4]{./images/logo_ensg}
    	\end{minipage}
    	&
    	%title box
    	\begin{minipage}[b][.1\textheight][b]{.6\textwidth}
    	
    		\centering
    		\color{titlefgcolor}
    		{\bfseries \Huge \sc \@title \par}
    		
    		{\bfseries \huge \sc Collaborative dashboard to study periurban densification \par}
    		\vspace*{2em}
    		{\huge \@author \par}
    		\vspace*{1em}
    		{\LARGE \@institute}
    		
    	\end{minipage}
    	&
    	%logo box 2
    	\begin{minipage}[b][.1\textheight][b]{.15\textwidth}
        	\begin{tabular}{c}
            	\includegraphics[width=1\textwidth]{./images/Logo_Subdense}\\~\\
        	\end{tabular}
    	\end{minipage}
	\end{tabular}
}
\makeatother

%%%%%%%%%%%%%%%%%%%%%%%%%%%%%%%%%%%%%%%%%%%%%%%%%%%%%%%%%%%%%%%%%%%%%%%%%%%%%%%%
% Multicol Settings
%%%%%%%%%%%%%%%%%%%%%%%%%%%%%%%%%%%%%%%%%%%%%%%%%%%%%%%%%%%%%%%%%%%%%%%%%%%%%%%%
\setlength{\columnsep}{1.5em}
\setlength{\columnseprule}{0mm}

%%%%%%%%%%%%%%%%%%%%%%%%%%%%%%%%%%%%%%%%%%%%%%%%%%%%%%%%%%%%%%%%%%%%%%%%%%%%%%%%
% Save space in lists. Use this after the opening of the list
%%%%%%%%%%%%%%%%%%%%%%%%%%%%%%%%%%%%%%%%%%%%%%%%%%%%%%%%%%%%%%%%%%%%%%%%%%%%%%%%


% To remove the "latex tikz poster" at the bottom right corner
\tikzposterlatexaffectionproofoff

\begin{document}
	% Title block with title, author, logo, etc.
	\maketitle
	
%------------------------------------------------------
%----------------LINE 1--------------------------------
%------------------------------------------------------
	\begin{columns}
		
		% first column with relative width
		% be sure that on a same line column widths sum to 1, otherwise columns are left aligned and it is awful
		
		\column{0.33}% Width set relative to text width
		
		% This is a block
		% The size of the block-title is specified with "titlewidthscale" and the name of the block is given after
		% The full formalism is \block[options]{title}{content}
		
		%==============================================
		\block[titlewidthscale=0.35]{Context}{
			\begin{itemize}
				\item Item 1
				\item Item 2
				\item Item 3 
				\item Item 4 
			\end{itemize}
		}
		%==============================================
		
		% Second column on same line than first one
		\column{0.34}
	    %==============================================
		\block[titlewidthscale=0.3,bodyoffsety=1.8cm,titleoffsety=.1cm]{Objectifs}{
			
			Texte pour expliquer les objectifs du travail 
			
			\vspace{.5cm}
		}
		%==============================================
		
		% Third column on first line
		\column{0.33}% Width set relative to text width
		%==============================================
		\block[titlewidthscale=0.35]{Méthode}{
			\begin{enumerate}
				\item Méthode 1
				\item Méthode 2
				\item Méthode 3
			\end{enumerate}
		}
		%==============================================
		%end of first line
	\end{columns}
	
%------------------------------------------------------
%----------------LINE 2--------------------------------
%------------------------------------------------------

	\begin{columns}
		
		\column{0.25}
		%==============================================
		\block[titlewidthscale=0.8]{Bloc 1}{
			Texte pour le bloc 1 


   
		}
		
		\column{0.45}
		%==============================================
		\block[titlewidthscale=0.52]{Bloc 2}{
		

   
		}
		%==============================================
		
		
		\column{0.3}
		%==============================================
		\block[titlewidthscale=0.95]{Bloc 3}{
		
		    %This block contain example of images
			\begin{center}
				%\includegraphics[scale=1.5]{images/}
	        \end{center} 
		}
		%==============================================
	\end{columns}
	
%------------------------------------------------------
%----------------LINE 3--------------------------------
%------------------------------------------------------

	% block alone on its line so you don't have to specify the columns environment

	%==============================================
	\block[titlewidthscale=0.12]{Résultats}{
		\parbox{0.5\textwidth}{
			%\includegraphics[width=18cm,height=20.5cm]{images/} \hspace{.5cm}
			%\includegraphics[width=18cm,height=20.5cm]{images/} \hspace{.5cm}
		}
		\parbox{0.35\textwidth}{
			
			
		}
		%==============================================
	}
	
%------------------------------------------------------
%----------------LINE 4--------------------------------
%------------------------------------------------------

	\begin{columns}
		%In this block you can specify possible improvment or any extra information
		\column{0.25}
		%==============================================
		\block[titlewidthscale=0.65]{Bloc 4}{
			\begin{itemize}
			%Here is example for cite a reference from your biblio. 
				\item Item 1 \citep{Biscotti15}
				\item Item 2 \citep{karpinski}
			\end{itemize}
		}
		
		\column{0.25}
		%==============================================
		\block[titlewidthscale=0.6]{Informations}{
			\begin{itemize}
				\item Date de début du projet 
				\item Institution
				\item Équipe
				\item Autre information
				\item Autre information
			\end{itemize}
		}
		
		\column{0.5}
		%==============================================
		% bibliography managed with natbib
		\block[titlewidthscale=0.3,bodyoffsety=1.8cm,titleoffsety=.1cm]{Bibliographie}{
			% This command is to prevent the printing of a second "références" in text and to delete white space between block title and bibliography
			\renewcommand\refname{\vskip -3.4cm}
			% Bibliography input. 
			%Refrences should be added directly to the Biblio.bib file. Refrences should in bibTex Format. Every Reference cited in the poster will be automatically added.
			
			\bibliography{Biblio}
			% Plain style so that cited articles appear as number in the poster and are only fully displayed here
			\bibliographystyle{plain}
		}
	    %==============================================	
	\end{columns}
	
\node [above right,outer sep=20pt,minimum width=\textwidth,align=center,draw=none,fill=none, text = IGNGrisFonce] at (bottomleft) {\centering \huge \bf Journées de la Recherche IGN - 2023};

\end{document}

\endinput



% Template


			\vspace{.5cm}
			
			%This block contain example of Tikz
			
			\centering
			
			\begin{tikzpicture}[scale=2]
			\draw[-latex] (0,0) -- (0,3.2);
			\draw[-latex] (0,0) -- (5.2,0);
			\node at (-0.2,1.5) {$\theta$};
			\node at (2.5,-0.2) {$\mathit{t}$};
			
			\node at (0.2,0) {$\bullet$};
			\node at (0.4,0.8) {$\bullet$};
			\node at (0.6,1.6) {$\bullet$};
			\node at (0.8,2.4) {$\bullet$};
			\node at (1.8,0.4) {$\bullet$};
			\node at (2,1.2) {\textcolor{red!60!black}{$\bullet$}};
			\node at (2.2,2) {$\bullet$};
			\node at (2.4,2.8) {$\bullet$};
			\node at (3.4,0) {$\bullet$};
			\node at (3.6,0.8) {$\bullet$};
			\node at (3.8,1.6) {$\bullet$};
			\node at (4,2.4) {$\bullet$};
			
			\draw[->,red!60!black] (2,1.2) -- (0.4,0.8);
			\draw[->,red!60!black] (2,1.2) -- (0.6,1.6);
			\draw[->,red!60!black] (2,1.2) -- (1.8,0.4);
			\draw[->,red!60!black] (2,1.2) -- (2.2,2);
			\draw[->,red!60!black] (2,1.2) -- (3.6,0.8);
			\draw[->,red!60!black] (2,1.2) -- (3.8,1.6);
			\end{tikzpicture}
			
			Exemple de Tikz 

			\vspace{.5cm}
			


